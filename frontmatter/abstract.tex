\thesisImprintTitle\\
\thesisImprintSubtitle\\
\thesisAuthor\\
\thesisDepartment\\
\thesisUniversity\setlength{\parskip}{0.5cm}

\thispagestyle{plain}			% Suppress header
\section*{\abstractname}

Insight into how biological sex and healthy ageing affects the human brain are important for an increased understanding of the brain. Healthy ageing insights are also useful for clinical applications, for instance in identifying unhealthy ageing due to neurodegenerative disease. To this end, several studies in the last few years have used machine learning methods on neuroscientific data to predict subject sex and brain age. One particularly interesting approach has been to represent functionally connected networks in the brain as graphs, and apply Graph Convolutional Networks (GCNs). To investigate which functional brain networks are connected with sex and age, we develop and analyse GCN-based models that predict sex and age from resting-state fMRI data. The analysis of the models is done using saliency mapping techniques that give insight into which functional brain networks in the data are relevant for the predictions. With this approach, we obtain a sex prediction accuracy of up to $79\,\%$ and an age prediction MAE of $5.9\rm\,years$. Furthermore, we find indications that the Somatomotor Medial Network and the cerebellum are among the more important functional brain networks for predicting sex and brain age.

\vfill
\if\thesisType M
    \textbf{Keywords:}
\else
    \textbf{Nyckelord:}
\fi
% lorem, ipsum, dolor, sit, amet, consectetur, adipisicing, elit, sed, do.
machine learning, supervised learning, GNN, GCN, explainability in AI, graph theory, population graphs, brain age, sex, functional connectivity, resting-state fMRI, saliency mapping.
\if\thesisLayout 2
\newpage				% Create blank page
\thispagestyle{empty}
\mbox{}
\fi
