\thesisImprintTitle\\
\thesisImprintSubtitle\\
\thesisAuthor\\
\thesisDepartment\\
\thesisUniversity\setlength{\parskip}{0.5cm}

\thispagestyle{plain}			% Suppress header
\section*{\abstractname}

Insight into how biological sex and healthy ageing affects the human brain is important, both for an increased understanding of the brain but also for possible clinical applications, for instance in identifying unhealthy ageing due to neurodegenerative disease. To this end, several studies have in the last few years used machine learning methods on neuroscientific data to predict subject sex and brain age. One particularly interesting approach has been to represent functionally connected networks in the brain mapped using resting-state fMRI as graphs, and apply Graph Convolutional Networks (GCNs). To investigate which functional brain networks are connected with sex and age, we develop GCN-based models for predicting sex and age using brain graphs derived from fMRI data. These models are analysed using saliency mapping techniques to gain insight into what functional brain networks are relevant for the predictions. With this approach, we obtain a sex prediction accuracy of up to $79\,\%$ and an age prediction MAE of $\sim5.9$ years. Furthermore, we find evidence that the Sensory Motor Network (SMN) and the cerebellum are among the more important functional brain networks for predicting sex and brain age.


\vfill
\if\thesisType M
    \textbf{Keywords:}
\else
    \textbf{Nyckelord:}
\fi
% lorem, ipsum, dolor, sit, amet, consectetur, adipisicing, elit, sed, do.
machine learning, supervised learning, GNN, GCN, explainability in AI, graph theory, population graphs, brain age, sex, functional connectivity, resting-state fMRI, saliency map.
\if\thesisLayout 2
\newpage				% Create blank page
\thispagestyle{empty}
\mbox{}
\fi
