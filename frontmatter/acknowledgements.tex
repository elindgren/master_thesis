\thispagestyle{plain}			% Suppress header
\section*{Acknowledgements}
This Master Thesis project is a collaboration between Syntronic AB, University of Gothenburg and Karolinska Institutet, and hence there are a number of people for which some thanks are in order. First of all, we would like to thank our supervisors, Alice Deimante Neimantaite and Lisa Sjöblom at Syntronic Research and Development AB. Alice and Lisa, your expertise, both in the domains of machine learning and in neuroscience, your support, and all the interesting discussions we have had, have been the cornerstone of this project. Next, our thanks go out to Joana Pereira, assistant professor, and Mite Mijalkov, post-doctoral researcher, at Karolinska Institutet (KI) for providing us with the data used in this thesis, as well as helping us interpret the results from a neuroscientific perspective. We would also like to thank Giovanni Volpe, professor at the Physics Department at University of Gothenburg, for providing guidance to the work and for being the examiner for this project. Finally, we would like to thank Syntronic AB for providing the opportunity for this fantastic collaboration, and for making the master thesis project possible in the first place.


\hfill
\thesisAuthor, \thesisCity, \thesisMonth\ \thesisYear
\vspace{0.5cm}
\section*{Declaration of contributions}
The data used in this thesis has been procured from the UK Biobank project, \url{https://www.ukbiobank.ac.uk/}, an extensive biomedical database for public health research containing data from over half a million participating subjects. Examples of types of data that are included in the biobank are subject sex, age, cognitive abilities, disease history, but also more specific information such as MRI and fMRI brain scans. Mite Mijalkov, post-doctoral researcher at KI, has provided us with the 21-node fMRI brain graphs for the approximately 35000 subjects used in this thesis, after which we have performed minimal preprocessing and stratification. Joana Pereira, assistant professor at KI, has contributed with the information in \cref{tab:Networks}, in which each of the 21 nodes are mapped to their corresponding functional brain network. Finally, \cref{fig:fmri_network} has been adapted from a figure from UK Biobank Brain Imaging Online Resources, \url{https://www.fmrib.ox.ac.uk/ukbiobank/}, specifically at \url{https://www.fmrib.ox.ac.uk/ukbiobank/netjs_d25/}. 

\if\thesisLayout 2
\newpage				% Create blank page
\thispagestyle{empty}
\mbox{}
\fi
