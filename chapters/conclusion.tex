% Conclusions
    % - Accuracy/MSE för modeller - första delen av syftet
    % - Noder har analyserats, vissa funktionella nätverk kanske viktigare än andra men alla viktiga
    % - 21 noders data är för lite men har sina fördelar

% Future work
    % - Andra modeltyper
    % - Annat similarity measure
    % - Undersök mer högupplöst data
    % - Inkludera andra modaliteter av data (strukturell MRI etc.)
    % - Flera analysmetoder (Grad-CAM, öppna upp modellerna mer)
    % - Försök förstå varför just de här noderna är viktiga, inte bara att de är det. Vad händer egentligen i hjärnan?

\chapter{Conclusion and Outlook}

In this study, accurate models based on GCNs for sex and brain age prediction using fMRI data have been developed and analysed with the goal of identifying functional brain networks that are related to sex and age. Three of the four studied models achieved comparable performance for both prediction tasks, with an accuracy of $\sim79\pm1\%$ for sex prediction a mean absolute error of $\sim5.9\pm0.1$ years for age prediction. From the node importance analysis it was concluded that all functional brain networks are important for sex and brain age prediction, but that some where slightly more important than others. Specifically, the Sensory Motor Network (SMN) and the cerebellum where indicated to be somewhat more important than other networks. The main limitation of the work was identified to be the low resolution of the fMRI data.

There are several improvements that could be considered in future works. One possible improvement is to study the use of more highly resolved fMRI data, for instance investigating the tradeoff between model performance and analysis interpretability. Similarly, the inclusion of other data modalities, such as using both structural MRI and fMRI data, could be investigated. It is possible that such combined approaches could improve model performance. Different model types, not based on fully connected or GCNs, could also be investigated. An example of such model types are various gradient-tree boosting algorithms, which have been previously applied to predicting brain age \cite{kaufmann}.

Studying different kinds of similarity measure could also be a possible future work. Two interesting approaches are graph based similarity measures such as in \cite{higcn}, or similarity measures based on machine learning to ``learn'' a measure, or using the results from node analysis to create a more educated measure.

% In the context of this thesis, other data than fMRI could for instance be used to develop a more information rich similarity measure for the population graphs. 

Other model analysis methods for determining node importance could also be studied. In this work, we have mainly utilized two methods, a naive node removal method and a masking method known as Zorro. Both of these methods perform input-output analysis of the models. To further investigate the explainability of the models, one could consider using approaches that ``open-up'' the neural networks. In Yuan et al. \cite{yuan_survey} a review over possible analysis methods for GCNs are presented. One example is Grad-CAM \cite{gradcam}, a method originally developed for analysing the gradients within CNNs to draw conclusion on the importance of each pixel in the input image for predicting a certain class. Grad-CAM has been generalised to GCNs, and was for instance used in both Arslan et al. \cite{arslan} and Kim and Ye \cite{understanding_gnn} to study the importance of functional brain networks for sex classification. We considered using Grad-CAM in this work, but weren't able to due to time limitations. 

This thesis has mainly focused on gaining insight into which functional brain networks are important for predicting sex and brain age, without approaching the question of why they are important. Possible future work could thus include studying these brain networks from a purely neuroscientific perspective, but one could also perform more detailed machine learning analysis. For instance, one could perform model analysis as in this work for models trained on only subsets of the dataset, to investigate which brain networks are important for e.g. predicting age in young or old subjects respectively. More detailed machine learning analysis could also help shed light on the interaction between the functional brain networks. Since one of the key takeaways from this thesis is that all nodes are important for predicting sex and age, it is reasonable to suspect there is some change in the complex interplay between the functional brain networks that is indicative of a subject's sex or brain age.
