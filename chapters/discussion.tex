\chapter{Discussion}
\label{chap:discussion}
% diskutera resultat för de olika modellerna
    % stat major findinga: more complex model not beneficial, graph structure no information and similarity measure is not good enough. 
    % Vad betyder findings (för låg dimensionel data, sim meas för simpelt/inte korrelerat med age) -- importance (kanske är bra att enkla modeller presterar bra)
    % Väv in alternative tolkningar -- grafstrukteren kanske är viktig, men ger inte högre prestanda, modellen inte generell.
    % Relatera till andra studier -- indikerar att våra modeller är begränsade av datans upplösning
    
Our results indicate two main findings with regards to the model performance. First, there seems to be no benefit to using more complex models than the Baseline regression model in general, and secondly the similarity measure does not seem to be effective in introducing extra information that improves the predictions for sex and brain age for the population graph models. 

% Each node is in itself an aggregation of multiple regions whose activations are highly correlated in time. Thus, the actual brain graphs that are fed into the models are separated from the original time-series data for the activation throughout the brain by several levels of abstraction.

A possible explanation as to why there is no benefit in using more complex models could be that the data is too low-resolved. The brain graphs that have been used in this study corresponds to networks of 21 nodes, where each node represents a functional network in the brain.  That the performance could be improved by using more highly resolved fMRI data is supported by similar works in the field, which reach higher predictive performance using similar models but with more highly-resolved data. Two examples of these for sex prediction are Arslan et al. \cite{arslan} and Kim and Ye \cite{understanding_gnn}. Kim and Ye reaches a sex classification accuracy of $84.61\pm2.9\%$ for a GNN known as a Graph Isomorphism Network (GIN), using brain graphs of 400 nodes where each node corresponds to a region of the brain. Arslan et al. also uses fMRI data from the UK Biobank, but aggregated into a network of 55 nodes, with which they reach a sex classification accuracy of $88.06\pm1.57\%$. These accuracies are higher than our results of $79.2\pm0.9\%$ accuracy for sex classification for a GCN, which indicates that our model performance could possibly be improved by using more highly-resolved data. This would likely improve the model performance for age prediction as well, where for example Stankevicieute et al. \cite{stankeviciute} obtains a mean squared error of $28.045\pm0.595$ years$^2$ and Amoroso et al. \cite{amoroso_multiplex_age} a mean absolute error of $4.7\pm0.1$ years. However, both of these studies utilise structural MRI data rather than fMRI data, and hence direct comparison is not possible. Another possibility is that better model performance could be achieved by investigating other model types, since all of the more complex models were based on graph convolutional layers, but that is outside the scope of this study. 

Low resolution of the 21 node data could also be a reason as to why there is no benefit in using GCNs specifically over Baseline. It is possible that the 21 node graphs are too small to exhibit complex graph structures that the GCN extracts information from, and that it thus focuses solely on the connections. This is supported by the fact that most applications of GCNs usually involve larger graphs. Examples of this includes Arslan et al. and Kim and Ye mentioned above, but also Kipf et al. \cite{kipf_semi_supervised}, the paper that originally introduces GCNs, which performs node prediction in citation networks with several thousand nodes. It is also possible that the graph structure in the 21 node graphs are informative for the GCNs, but that the graphs are so small that the Baseline regression model also manages to extract the same information from the graph structure. However, the number of parameters in Baseline and other fully connected networks grow very rapidly with the size of the graph, which makes them impractical for large graphs. In the case of either explanation, it is possible that the performance of the GCNs could increase by using more highly resolved graph data.


When the models for the thesis was developed the way to increase the performance of the GCN model was to introduce the population graph. That the increase in performance where not observed for Popencoder was interpreted as the similarity measure was not effective in adding more information. To understand why lets again consider the heuristic explanation of why the population graph would be beneficial. The heuristic explanation was that an initial prediction of sex or age could be corrected by considering more or less complex structures of similarities in the data set. That this would help is based on the assumption that people with high similarity scores, would have approximately the same age. This absent of performance increase thus suggests that the similarity measure can not indicate if two subjects have similar age or sex, and the need for better similarity measure is eminent. Other approaches for a similarity measure is used in \cite{stankeviciute} and \cite{higcn}, which uses non-imaging data such as education and cognitive abilities, and a graph theoretical motivated similarity measure respectively. Both these approaches however seems to have the same problem: the similarity measure is not correlated with the predictions. One way to make a the similarity measure better is to include more domain now ledge. Domain knowledge could for example be knowledge about which functional brain networks are related sex and age which is part of the analysis in this thesis, but other types of domain knowledge is important. Another idea to make the similarity measure better is to learn it with machine learning. Constructing similarity measures from data is for example done in \cite{}.


% The fact that the similarity measure is ineffective in introducing extra information exemplifies how crucial it is to develop an appropriate similarity measure for population graphs to be effective. The idea behind the population graph approach was that the model predictions for each subject could be corrected by introducing information on similarities into the models, thus increasing model performance. Other approaches for designing the similarity measure include Stankeviceute et al. mentioned above, which use non-imaging data that may be related to brain age to form the similarity measure, and  which use a graph theoretically motivated similarity measure. Thoroughly studying different similarity measures for age and sex prediction was outside of the scope for this study, but it is something that would be interesting to investigate in a future work.



% Another approach for designing the similarity measure is to include other types of information for each subject. For example, in Stankeviceute et al. mentioned above, the similarity between subjects is calculated based on other information than brain imaging data that may be related to brain age, with the imaging data being used as features for the models. Such an approach was not attempted in this study, since selecting what data to use for the similarity measure requires extensive domain knowledge, and because the aim of the thesis was to study fMRI data for sex and brain age prediction. However, it would be interesting to investigate in the future.



% diskutera analys resultaten
    % Major findings: 
        % - alla noder är to some extent important men Ingen är jätte viktig. (ingen random guessing om någon tas bart)
        %     - Hela hjärnan påverkas när man åldras/kön, verkar inte vara så "enkelt" som ett nätverk som styr. 
        % - vissa verkar var mer viktiga än andra (relativt, kanske inte absolut)
        %     - Nämn specifikt vissa nätverk (inte noder)     
    % Nytt stycke, diskutera kring metoderna
        % - Verkar vara skillnader mellan GCN/FFNN 
        %     - Det kanske finns något i grafstrukturen
        %     - GCN är mer robust mot brus i indatan
    % explain meaning of the findings
        % - Varför ger zorro olika resultat för basline och gcn. 
        % - Varför får vi lite olika resultat för niave och zorro.
    % Relate to other findings 
        % - Relatera metod
        % - Relatera resultat
    % Limitations
        % Analysen bygger på modellresultaten, begränsningar i e.g. lågupplöst data propagerar
        % Kan finnas fler
        % Fördelar: 80% ändå ganska bra, 21 nodersdata är mer lättolkad
% Övergripande diskussion - jämför modellresultat & analysresultat med andra papers


%%% New idea 
    % major findings
        % - for sex class node 12 seems to be most important but also nodes ... have some indications of higher imortance
        % - for age prediction node 12 and 15 seems to be most important but other nodes with mixed results may be important
        % - Zorro seems to give slightly different results for baseline and GCN
        % - The two analysis methods are generally in agreement with some differences (in addition to the spikes for baseline)
        
        % ... importance of validation (Zorro and naive) and Zorro might give higher importance than necessary \implicerar node 12 and 15 stands out. 


The result from the analysis part of the thesis states that all functional brain networks are to some extent important but that no single node is crucial for predicting sex and age. The result also shows significant differences in how important different nodes are but with some inconsistencies. More precisely, the result from Zorro when analysing Baseline and GCN are inconsistent and Baseline gets generally higher importance scores, otherwise the two analysis methods are roughly in agreement. 


That all nodes are important for the predictions was expected since it is reasonable that the hole brain is effected when for example ageing. That some nodes are more important is interpreted that some functional networks or at least some part of the functional network are to higher degrees connected to differences in sex and age. That the result for the two methods and the two models are roughly in agreement is very reassuring since it gives strength to the conclusions. The inconsistencies observed is also bound to a higher importance score of Baseline for node 1,9 and 14 in the age prediction case and node 8 in the sex prediction case. These inconsistency are very interesting and to be able to conclude which node is most important they need to be discussed. 

The higher importance score of Baseline means that more unmasked nodes are needed before the Zorro algorithm stops adding nodes to the explanations, i.e. the fidelity of the explanation is higher then $\tau = 0.9$. The fidelity indicates how likely a model is to give the same prediction as it's original one when only the nodes according to the explanation is not replaced by noise. The higher importance scores of Baseline can thus be interpreted as GCN model are more robust in the sense it is not as susceptible to noise.
This indicates that it might be some differences between the two model, even if it is augmented not to be in \ref{sec:}???. An explanation for how this  is possible can be that the previous discussion argues that the model are the same in the sense that the same information is extracted and their performance is the same. The differences between the models in the analysis may indicate that although they utilize the same information and have the same performance the way they make prediction is slightly different. GCN perhaps gets a more robust prediction not as susceptible to noise because it can formulate the same information Baseline can in the context of graph structure. 

Also note that an increased number of nodes in the explanation can be the result of changing the hyper parameter $\tau$. Increasing the fidelity required to stop adding nodes $\tau$ for GCN means that the result for Baseline and GCN might become consistent. Also note that if $\tau$ is chosen too high all nodes will probably need to be included in the explanations, which will give no differentiation on which node are important. This enable at least two interpretations, the first one being that node 1, 9 and 14 for age and 8 for sex arises since $\tau$ was chosen to high. These nodes may hence  not really be important for age or sex, but just happens to be important for Baseline to make these specific predictions, and arises since Baseline possibly is more sensitive to noise. For the other interpretation consider tuning the hyper parameter $\tau$, more and more nodes will get added as $\tau$ increase. Node 1, 9, 14, and 8 may have been the last nodes to be added to the explanations indicating that they where deemed important for this level of $\tau$ and hence 12 and 15 may be more important. Both of these explanations would indicate that node 1, 9, 14, and 8 may not be as important for age and sex as they may seem to be. However to be sure of how to interpret the result from Zorro both for Baseline and GCN other values of the hyper parameter $\tau$ would need to be investigated, which was not done due to time limitations. 

With the inconsistencies regarding the Zorro results discussed the small but nevertheless present differences between the naive node masking and Zorro methods needs to be addressed. First note that since $\tau$ may change the result of Zorro another choice of $\tau$ could have made the result even more similar. Even so it is still important to remember that the two methods answers slightly different questions. Naive node masking answers the question which removed node effect the performance of the models the most, whilst Zorro answers the question which node is most important to unmask for a prediction to be as close to the original as possible. Even if the two questions seem to be very similar the result should not be expected to be identical. For example, consider a model that is overfit to the training data, and is erronously basing its predictions on a certain node. In the case of Zorro, removing that node will have a large impact since the model is dependent on it for making predictions. In the case of naive node masking, removing the node and retraining the model may give the model an opportunity to adapt and instead base the predictions on other nodes, and the removed node would be interpreted to be less important. In this manner, differences between the methods may appear. However, the underlying assumption for the analysis is that removing or replacing an important node will remove so much important information for predicting sex and age that both methods will reliably identify the change. Thus, both methods answer the main question in this thesis on the importance of each node for predicting sex and age. The fact that the methods answer slightly different questions could even be a strength in the sense that it increases the credibility of the results, since the analysis methods are different but still yield comparable results. Extending the analysis to include other methods could thus be beneficial. This was unfortunately not possible in this thesis though due to time constraints, and is left as future work.

So with the inconsistencies in Zorro and the differences between Zorro and the naive method discussed the result can be re-examined in this new light. The discussion about Zorro made the need for validating against naive evident. Comparing the methods further strengthen the interpretation that node 8 is not important for sex and node 1, 9 and 14 is not important for age. Furthermore cross validation states that node 12 is most important for sex classification with node 15 and 16 being the runner ups. For age prediction comparing the two methods for both models clearly indicates that node 12 and 15 is the most important nodes and that node 20 also may be somewhat important. 



In \cite{understanding_gnn} which used similar models for performing sex classification the saliency mapping generally indicated that the SMN and DMN were the most important. Also in \cite{arslan} the DMN was deemed important for sex classification. In both these articles the saliency mapping was conducted  by investigating gradients in the model, which is an alternative approach to the methods used in this thesis. Comparing the results is very interesting since our findings is in agreement with \cite{understanding_gnn} in the sense that SMN seems to be important, but it is in disagreement with both \cite{gin, arslan} because we don't find the DMN to be as important. 

An explanation to the disagreement can be because of the different methods used to perform the salience mapping. Just as for the naive node masking method and Zorro the methods in \cite{understanding_gnn, arslan} might answer different question then our methods and thus gives different results. This highlight the complexity of understanding machine learning models, and simply stating that a node is important might be insufficient. To be sure a specific network is related to sex or age the important question to answers is why is this network important for the prediction. 

Another possible explanation for the difference in \cite{understanding_gnn, arslan} can be that they uses more highly resolved data, which previously has been regarded as an explanation of the lower performance of our models. For example \cite{arslan} identifies four important brain region for the prediction, which all resides in the DMN, but in our data DMN constitutes one node. The connection between the regions in the DMN or their individual connection to other brain regions may be important for the classification, which can explain the lower performance of our models. It would also explain why the DMN is not deemed important from the analysis made in this thesis, since the information important for sex classification that resided in the DMN is not present in our data.

% Age
Other studies have also investigated functional brain networks related to ageing. In Song et al. \cite{song_reorganizational} graph theoretical analysis was used to conclude that the DMN and the SMN where found to undergo relatively large changes in connectivity with age. Tomasi et al. \cite{tomasi_aging} studied the connections between networks and found that the DMN, the Dorsal Attention Network (DAN), the Somatosensory Network (SSN), cerebellum, thalamus and amygdala exhibited pronounced changes in functional connectivity with age. A machine learning approach using SVMs was used in Meier et al. \cite{meier_svm}, where the DMN, SMN and cingulo-opercular networks where found to be related to age. 


% Mer jämförelser med andra resultat. Måste inte vara ML utan bara nått som sett att vissa nätverk verkar vara kopllade till kön eller ålder.




Since the analysis methods used in this thesis are based on investigating the machine learning models, the siginificance of the analysis results is limited by the performance of the models. The underlying assumption for the analysis to be meaningful is that the machine learning models can extract relevant information for predicting sex and brain age from the data. Thus, any limitations on the model performance is by extension a limitation on the analysis. In this sense, the low resolution of the data used to train the machine learning models hinders the analysis. Because of this, using more highly resolved data is crucial for improving the significance and making the analysis more detailed. However, using higher dimensionality data can make the analysis technically more challenging and interpretation of the results more difficult. From this perspective, one could argue that a decrease in model performance is worth it for the sake of interpretability of the results. Interpretability is important both for research purposes, but also for clinical applications. Specifically, the 21 node data used in this thesis has the benefit of each node roughly corresponding to a functional brain network, which greatly simplifies the interpretation of the analysis. Another benefit of using the 21 node data is that, as our results indicate, the use of more complex models is not necessary. Using simpler models, such as the Baseline regression model, makes the models themselves more explainable. Thus, there exists a clear trade-off between model and analysis performance and interpretability. Finding the sweet spot between these two is key to further gain insight into how functional brain networks are related to sex and ageing using machine learning. 


% On the other hand, there are several benefits of using the more highly abstracted 21 node brain graphs, with the most prominent one being ease of interpretation, since each node corresponds to a functional brain network. From this perspective, one could argue that a trade-off in model performance is worth it for the sake of interpretability of the results. Interpretability is important both for research purposes, but also for clinical applications. Another benefit of using the 21 node data is that, as our results indicate, the use of more complex models is not necessary. Using simpler models, such as the Baseline regression model used in this study, makes the models themselves more explainable, for more of a white-box approach than the traditional black-box approach that are neural networks. The combination of using abstracted data with simple models that doesn't reach state-of-the-art performance but are inherently more interpretable, thus have the benefit of improving the transparency of the sex and age prediction procedures, which simplifies understanding and analysis of models and data. 

% In summary, the main limitations of the model performance can thus be argued to be the low resolution of the 21 node brain graphs. This is a both a direct limitation in the model performance, but also for the model analysis which will be presented next. The aim of the analysis is to determine important functional brain networks for sex and age prediction, and is hence based on the assumption that the models that are analysed are accurate and can extract relevant information for predicting sex and age. There is thus a direct trade-off between model performance and interpretability, where the models have to perform well enough to be analysed, but still not be too complex and use simple data as to be easily interpretable.


% Since the analysis methods of this thesis is based on investigating the machine learning models the significance of the result is limited by the performance of the models. If a model performance poorly functional brain network deemed important might not be important at all. mening här om att modellerna ändå är tillräckligt bra för att ge indikationer. Since the models performance is limited because of the dimensionallity of the data also the analysis is. The attitude towards the 21 node data is from the analysis perspective twofolded. As pointed out the low dimensionallity inhibits the significance due to worse model performance. The analysis however in a sens becomes easier to do since each node corresponds to a single functional brain network. 















