\chapter{Discussion}
\label{chap:discussion}
% diskutera resultat för de olika modellerna
    % stat major findinga: more complex model not beneficial, graph structure no information and similarity measure is not good enough. 
    % Vad betyder findings (för låg dimensionel data, sim meas för simpelt/inte korrelerat med age) -- importance (kanske är bra att enkla modeller presterar bra)
    % Väv in alternative tolkningar -- grafstrukteren kanske är viktig, men ger inte högre prestanda, modellen inte generell.
    % Relatera till andra studier -- indikerar att våra modeller är begränsade av datans upplösning
    
Our results indicate two main findings with regards to the model performance. First, there seems to be no benefit to using more complex models than the Baseline regression model in general, and secondly the similarity measure does not seem to introduce enough information to improve the predictions of sex and brain age.

% Low resolution of the 21 node data could also be a reason as to why there is no benefit in using GCNs specifically over Baseline.

% It is possible that the 21 node graphs are too small to exhibit complex graph structures that the GCN extracts information from, and that it thus focuses solely on the connections. This is supported by the fact that most applications of GCNs usually involve larger graphs. Examples of this includes Arslan et al. and Kim and Ye mentioned above, but also Kipf et al. \cite{kipf_semi_supervised}, the paper that originally introduces GCNs, which performs node prediction in citation networks with several thousand nodes.
%  The brain graphs that have been used in this study corresponds to graphs of 21 nodes, where each node represents a functional network in the brain.
A possible explanation as to why there is no benefit in using more complex models over Baseline could be that the data is too low-resolved. The 21 node brain graphs used in this study could be so small that the Baseline regression model also manages to extract the same information from the graph structure. The number of parameters in Baseline and other fully connected networks grow very rapidly with the size of the graph, which makes them impractical for large graphs, but in principle they should perform as well as a GCN for a smaller graph. In the case of either explanation, it is possible that the performance of the GCNs could increase by using more highly resolved graph data.

% A possible explanation as why there is no benefit in using more complex models over Baseline could be that the data is too low-resolved. In this sense, all models can extract all information from the graphs.  

% It is furthermore not surprising that Baseline performs as well as GCN. Theoretically, a deep fully connected neural network should be able to approximate any function. Baseline is not deep, but since the data is low dimensional it is not very surprising. 

That the performance could be improved by using more highly resolved fMRI data is further supported by similar works in the field, which reach higher predictive performance using similar models but with more highly-resolved data. For sex classification \cite{arslan, understanding_gnn} reaches classification accuracies of $84\%$ and $88\%$, with brain graphs of size 400 and 55 nodes respectively. This is likely the case for age prediction as well, where for example \cite{stankeviciute} obtains a MSE of $\sim28$ and \cite{amoroso_multiplex_age} a MAE of $\sim4.7$ years, both using high-dimensional MRI data. Note though that the results are not directly comparable to ours due to the use of MRI and not fMRI data.

The limitation of low data dimensionality motivated the introduction of the population graph, as an attempt at improving the performance of the GCN models. The heuristic explanation for why population graphs could be effective was that an initial prediction of sex or age could be corrected by considering more or less complex structures of similarities in the data set. That this would improve performance is based on the assumption that people with high similarity scores, would have approximately the same age or sex. The absence of a performance increase for Poptoy and Popencoder thus suggests that the similarity measure can not indicate if two subjects have similar age or sex, and the need for a better similarity measure is apparent. Other approaches for a similarity measure is used in \cite{stankeviciute} and \cite{higcn}, which uses non-imaging data such as education and cognitive abilities, and a graph theoretical motivated similarity measure respectively. Another way to make the similarity measure better is to include more domain knowledge. Domain knowledge could for example be knowledge about which functional brain networks are related to sex and age which is part of the analysis in this thesis, but other types of domain knowledge are important. High-performing similarity measure could also be learnt using machine learning, which enables complex and abstract measures which perhaps would be hard to find even for domain experts. A example of how to construct similarity measures from data is for example done in \cite{sim_meas_constr}. Investigating more sophisticated similarity measures was however not possible in this study and are left as future research. 


% diskutera analys resultaten
    % Major findings: 
        % - alla noder är to some extent important men Ingen är jätte viktig. (ingen random guessing om någon tas bart)
        %     - Hela hjärnan påverkas när man åldras/kön, verkar inte vara så "enkelt" som ett nätverk som styr. 
        % - vissa verkar var mer viktiga än andra (relativt, kanske inte absolut)
        %     - Nämn specifikt vissa nätverk (inte noder)     
    % Nytt stycke, diskutera kring metoderna
        % - Verkar vara skillnader mellan GCN/FFNN 
        %     - Det kanske finns något i grafstrukturen
        %     - GCN är mer robust mot brus i indatan
    % explain meaning of the findings
        % - Varför ger zorro olika resultat för basline och gcn. 
        % - Varför får vi lite olika resultat för niave och zorro.
    % Relate to other findings 
        % - Relatera metod
        % - Relatera resultat
    % Limitations
        % Analysen bygger på modellresultaten, begränsningar i e.g. lågupplöst data propagerar
        % Kan finnas fler
        % Fördelar: 80% ändå ganska bra, 21 nodersdata är mer lättolkad
% Övergripande diskussion - jämför modellresultat & analysresultat med andra papers


%%% New idea 
    % major findings
        % - for sex class node 12 seems to be most important but also nodes ... have some indications of higher imortance
        % - for age prediction node 12 and 15 seems to be most important but other nodes with mixed results may be important
        % - Zorro seems to give slightly different results for baseline and GCN
        % - The two analysis methods are generally in agreement with some differences (in addition to the spikes for baseline)
        
        % ... importance of validation (Zorro and naive) and Zorro might give higher importance than necessary \implicerar node 12 and 15 stands out. 


The result from the analysis part of the thesis states that all functional brain networks are to some extent important but that no single node is crucial for predicting sex or age. That all nodes are important for the predictions was expected since it is reasonable that the whole brain is affected during e.g. ageing. The result also shows significant variations in how important different nodes are. Since the data used in this thesis consists of connections between functional brain networks, the interpretation of a node being important is that its connections to all other functional networks are important for predicting sex or brain age. That some nodes are more important is thus interpreted that some functional networks, or at least parts of the functional networks, are to a higher degree related to differences in sex and age. Furthermore, the two analysis methods are generally in agreement, but there are some inconsistencies. More precisely, the result from Zorro when analysing Baseline and GCN are inconsistent where Baseline obtains higher importance scores for CB1, DMN and SN for age and VV1M for sex. These inconsistencies are very interesting and to be able to conclude which nodes are most important they need to be discussed. 

The higher importance score of Baseline means that more unmasked nodes are needed for the explanation to reach a fidelity of $\tau = 0.9$. The fidelity indicates how likely a model is to give the same prediction when only the nodes in the explanation are not replaced by noise. The higher importance scores of Baseline can thus be interpreted as the GCN model being more robust in the sense that it is not as susceptible to noise. This indicates that there might be some differences between the two models, even if their performance is roughly the same. The differences between the models in the analysis may indicate that although they utilize the same information and have the same performance the way they make prediction is slightly different. GCN perhaps obtains a more robust prediction not as susceptible to noise because it can formulate the same information Baseline can in the context of graph structure. 

An increased number of nodes in an explanation could also be obtained by changing the hyper parameter $\tau$. By increasing $\tau$ for GCN more nodes will thus be needed, and the result for Baseline and GCN might become consistent. Also note that if $\tau$ is chosen too high all nodes will probably need to be included in the explanations, which will give no differentiation on which nodes are important. This enable at least two interpretations, the first one being that CB1, DMN, SN and VV1M are not important for age and sex predictions respectively, but arises since $\tau$ was chosen too high. They are then only important for making specific predictions and may be artefacts of Baseline being more susceptible to noise. For the other interpretation, it's possible that CB1, DMN, SN and VV1M are important for predicting sex and age, but less important than SMM and CB2. Since Baseline requires more nodes to reach $\tau = 0.9$, it's possible that CB1, DMN, SN and VV1M thus were the last nodes added indicating that they are less important. Both of these explanations would indicate that CB1, DMN, SN and VV1M may not be as important for age and sex as they may seem to be. However to be sure of how to interpret the result from Zorro both for Baseline and GCN other values of the hyper parameter $\tau$ would need to be investigated, which was not done due to time limitations. 


Having discussed the inconsistencies regarding the Zorro results, the small but nevertheless present differences between the naive node masking and Zorro methods needs to be addressed. These differences may perhaps be reduced by another choice of $\tau$, since another $\tau$ affects the result of Zorro. Even so it is still important to remember that the two methods answers slightly different questions. Naive node masking answers the question which removed node effect the performance of the models the most, whilst Zorro answers the question which node is most important to unmask for a prediction to be as close to the original as possible. Even if the two questions seem to be very similar the result should not be expected to be identical. However, both questions still indicates how important nodes are for the predictions, and using two different method can thus strengthen the credibility of the results. Extending the analysis to include more methods could thus be beneficial, but this is left as future work.

With the two analysis methods discussed, we can now re-examine the analysis results in this new light. Comparing the methods further strengthens the interpretation that VV1M for sex and CB1, DMN and SN for age are not among the most important nodes. Furthermore, comparison yields that SMM is most important for sex classification followed by node CB2 and PL. For age prediction comparing the two methods for both models clearly indicates that node SMM and CB2 are the most important nodes and that node PMC also may be somewhat important. Interestingly, nodes SMM and CB2 are thus deemed important for both sex and age prediction. This overlap may be expected, since sex and age are covariates \cite{zhang_covariates}, in the sense that changes in functional connectivity of the brain that are related to age may also be related to sex. In this work we have focused on the effects of sex and ageing in general, without controlling for the other. This may thus be a limitation in the sense that the analysis e.g. compounds the effects of sex-related differences in functional connectivity when studying ageing and vice versa. More in-depth analysis in which sex and age are controlled for would thus be a suitable direction for future work.

Comparing our results with the literature, we find both similarities and inconsistencies. For sex prediction, \cite{understanding_gnn} performed a saliency mapping using similar machine learning models as in this thesis, and arrived at the DMN and SMN being the most important for sex prediction. In \cite{arslan}, the DMN was found to be important. For age prediction, Song et al. \cite{song_reorganizational} found the DMN and SMN to undergo relatively large changes in connectivity with age. Tomasi et al. \cite{tomasi_aging} studied the connections between networks and found that the DMN, the Dorsal Attention Network (DAN), the Somatosensory Network (SSN), cerebellum, thalamus and amygdala exhibited pronounced changes in functional connectivity with age. A machine learning approach using SVMs was used in Meier et al. \cite{meier_svm}, where the DMN, SMN and cingulo-opercular networks were found to be related to age. Most of these works are generally in agreement with our results of what networks are important for predicting sex and brain age. The SMN is often found to be among the most important networks for both sex and age, and the cerebellum is obtained in \cite{tomasi_aging}. However, one major inconsistency is that all of the studied articles find the DMN to be among the most important networks. 

One possible explanation for the inconsistencies regarding the DMN may be the dimensionality of the data. In several of the studies mentioned higher dimensional data have been used and the DMN and other network consist of several nodes. It is possible that the useful information for sex or age prediction resides within the DMN or in how individual parts of the DMN is connected to other networks. This could explain both the fact that we obtain slightly less performance than for example \cite{arslan, understanding_gnn} and why DMN is not considered as important as other networks in our thesis. It is still important to remember that we find all networks to be important including DMN. Several of the studies which have observed age related changes in DMN \cite{meier_svm, song_reorganizational, tomasi_aging} have no way of ranking the networks, so or observation of SMN and cerebellum being the most important do not need to be an inconsistency. In \cite{arslan, understanding_gnn} they however have the ability to rank imortance of nodes for sex classification and they find DMN to be one of the most important while not even  mentioning the cerebellum. An explanation to the disagreement can be because of the different methods used to perform the salience mapping. Just as for the naive node masking method and Zorro, the methods in \cite{understanding_gnn, arslan} might answer different question than our methods and thus gives different results. This highlight the complexity of understanding machine learning models, and simply stating that a node is important might be insufficient. To be sure a specific network is related to sex or age the important question to answer is why and how this network is important for the prediction. 


Since the analysis methods used in this thesis are based on investigating the machine learning models, the significance of the analysis results is limited by the performance of the models. The underlying assumption for the analysis to be meaningful is that the machine learning models can extract relevant information for predicting sex and brain age from the data. Thus, any limitations on the model performance is by extension a limitation on the analysis. In this sense, the low resolution of the data used to train the machine learning models hinders the analysis. Because of this, using more highly resolved data is crucial for improving the significance and making the analysis more detailed. However, using higher dimensionality data can make the analysis technically more challenging and interpretation of the results more difficult. From this perspective, one could argue that a decrease in model performance is worth it for the sake of interpretability of the results. Interpretability is important both for research purposes, but also for clinical applications. Specifically, the 21 node data used in this thesis has the benefit of each node roughly corresponding to a functional brain network, which greatly simplifies the interpretation of the analysis. Another benefit of using the 21 node data is that, as our results indicate, the use of more complex models is not necessary. Using simpler models, such as the Baseline regression model, makes the models themselves more explainable. Thus, there exists a clear trade-off between model and analysis performance and interpretability. Finding the sweet spot between these two is key to further gain insight into how functional brain networks are related to sex and ageing using machine learning. 
















