\chapter{Discussion}
\label{chap:discussion}
% diskutera resultat för de olika modellerna
    % stat major findinga: more complex model not beneficial, graph structure no information and similarity measure is not good enough. 
    % Vad betyder findings (för låg dimensionel data, sim meas för simpelt/inte korrelerat med age) -- importance (kanske är bra att enkla modeller presterar bra)
    % Väv in alternative tolkningar -- grafstrukteren kanske är viktig, men ger inte högre prestanda, modellen inte generell.
    % Relatera till andra studier -- indikerar att våra modeller är begränsade av datans upplösning
    
Our results indicate two main findings with regards to the model performance. First, there seems to be no benefit to using more complex models than the Baseline regression model, and secondly the similarity measure does not seem to introduce enough information to improve the predictions of sex and brain age.

A possible explanation as to why there is no benefit in using more complex models over Baseline could be that the data is too low-resolved. The simple Baseline model is perhaps able to extract all relevant information from the 21 node graphs, and hence the more complex models can't improve upon the performance of Baseline. Increasing the data dimensionality could thus potentially increase the performance for the models presented in this thesis. That the simpler Baseline model performs equally well as the more complex methods has advantages in the sense of explainability, since a simpler model is easier to understand and interpret. Explainability is especially important for potential clinical applications, where machine learning is applied to real patients. However, when the data dimensionality is increased, the number of parameters in Baseline scales poorly, and using the more complex models might be necessary. 

That the performance could be improved by using more highly resolved \acrshort{fmri} data is further supported by similar works in the field, which reach higher predictive performance using similar models but with more highly-resolved data. For sex classification \cite{arslan, understanding_gnn} reaches classification accuracies of $84\%$ and $88\%$, with brain graphs of size 400 and 55 nodes respectively. This is likely the case for age prediction as well, where for example \cite{stankeviciute} obtains a \acrshort{mse} of $\sim28$ and \cite{amoroso_multiplex_age} a \acrshort{mae} of $\sim4.7$ years, both using high-dimensional \acrshort{mri} data. Note though that the results are not directly comparable to ours due to the use of \acrshort{mri} and not \acrshort{fmri} data.

The limitation of low data dimensionality motivated the introduction of the population graph, as an attempt at improving the performance of the \acrshort{gcn} models. The heuristic explanation for why population graphs could be effective was that an initial prediction of sex or age could be corrected by considering more or less complex structures of similarities in the data set. That this would improve performance is based on the assumption that people with high similarity scores would have approximately the same age or sex. The absence of a performance increase for Poptoy and Popencoder thus suggests that the similarity measure can not indicate if two subjects have similar age or sex, and the need for a better similarity measure is apparent. Other approaches for a similarity measure is used in \cite{stankeviciute} and \cite{higcn}, which uses non-imaging data such as education and cognitive abilities, and a graph theoretical motivated similarity measure respectively. Another way to make the similarity measure better is to include more domain knowledge. Domain knowledge could for example be knowledge about which functional brain networks are related to sex and age, which is part of the analysis in this thesis, but other types of domain knowledge are important. High-performing similarity measure could also be learnt using machine learning, which enables complex and abstract measures, which perhaps would be hard to find even for domain experts. An example of how to construct similarity measures from data is given in \cite{sim_meas_constr}. Investigating more sophisticated similarity measures was however not possible in this study and are left as future research. 


% diskutera analys resultaten
    % Major findings: 
        % - alla noder är to some extent important men Ingen är jätte viktig. (ingen random guessing om någon tas bart)
        %     - Hela hjärnan påverkas när man åldras/kön, verkar inte vara så "enkelt" som ett nätverk som styr. 
        % - vissa verkar var mer viktiga än andra (relativt, kanske inte absolut)
        %     - Nämn specifikt vissa nätverk (inte noder)     
    % Nytt stycke, diskutera kring metoderna
        % - Verkar vara skillnader mellan GCN/FFNN 
        %     - Det kanske finns något i grafstrukturen
        %     - GCN är mer robust mot brus i indatan
    % explain meaning of the findings
        % - Varför ger zorro olika resultat för basline och gcn. 
        % - Varför får vi lite olika resultat för niave och zorro.
    % Relate to other findings 
        % - Relatera metod
        % - Relatera resultat
    % Limitations
        % Analysen bygger på modellresultaten, begränsningar i e.g. lågupplöst data propagerar
        % Kan finnas fler
        % Fördelar: 80% ändå ganska bra, 21 nodersdata är mer lättolkad
% Övergripande diskussion - jämför modellresultat & analysresultat med andra papers


%%% New idea 
    % major findings
        % - for sex class node 12 seems to be most important but also nodes ... have some indications of higher imortance
        % - for age prediction node 12 and 15 seems to be most important but other nodes with mixed results may be important
        % - Zorro seems to give slightly different results for baseline and GCN
        % - The two analysis methods are generally in agreement with some differences (in addition to the spikes for baseline)
        
        % ... importance of validation (Zorro and naive) and Zorro might give higher importance than necessary \implicerar node 12 and 15 stands out. 


The result from the saliency mapping analysis part of the thesis states that all functional brain networks are to some extent important, but that no single node is crucial for predicting sex or age. That all nodes are important for the predictions was expected since it is reasonable that the whole brain is affected during e.g. ageing. The result also shows significant variations in how important different nodes are. Since the data used in this thesis consists of connections between functional brain networks, the interpretation of a node being important is that its connections to all other functional networks are important for predicting sex or brain age. That some nodes are more important is thus interpreted that some functional networks, or at least parts of the functional networks, are to a higher degree related to differences in sex and age. Furthermore, the two analysis methods are generally in agreement, but there are some inconsistencies. More precisely, the result from Zorro when analysing Baseline and GCN are inconsistent where Baseline obtains higher importance scores for \acrshort{cb1}, \acrshort{dmn} and \acrshort{sn} for age and \acrshort{vv1m} for sex. These inconsistencies are very interesting and to be able to conclude which nodes are most important they need to be discussed. 


% An increased number of nodes in an explanation could also be obtained by changing the hyper parameter $\tau$. By increasing $\tau$ for GCN more nodes will be needed, and the result for Baseline and GCN might become consistent. Also note that if $\tau$ is chosen too high all nodes will probably need to be included in the explanations, which will give no differentiation on which nodes are important. This enables at least two interpretations, the first one being that \acrshort{cb1}, \acrshort{dmn}, \acrshort{sn} and \acrshort{vv1m} are not important for age and sex predictions respectively, but arises since $\tau$ was chosen too high. They are then only important for making specific predictions and may be artefacts of Baseline being more susceptible to noise. 


% Even if the result indicate differences between the two models it is important to remember that these results are only valid for $\tau = 0.9$. It is possible that the result for both Baseline and GCN is changed for other values of $\tau$. More specifically more nodes will probably be needed if $\tau$ was increased. So either the result indicates that the models find different information in the data, which is deemed unlikely due to thier similar performance or t

These inconsistencies may be interpreted as different functional networks are regarded as important for Baseline and GCN. Even if this is possible, it is deemed unlikely because of the similar performance of the two models. To explain the inconsistencies if both models finds the same functional networks to be important, it is imperative to realise that the obtained result is only valid for $\tau = 0.9$. Changing $\tau$ may affect the result for both Baseline and GCN, where increasing or decreasing $\tau$ would lead to more/fewer nodes being added respectively. This enables two possible interpretations; \acrshort{cb1}, \acrshort{dmn}, \acrshort{sn} and \acrshort{vv1m} are either not important for age and sex predictions, or they are important. In the case of the nodes not being important, $\tau$ could have been selected to high, meaning the nodes are included to recreate specific predictions rather than being indicative of sex or age. 
In the case of the nodes being important, it is possible that the nodes will be added to the explanations for GCN for a slightly higher value of $\tau$. In this scenario, the nodes could be regarded as less important than nodes \acrshort{smm} and \acrshort{cb2}, since they are added last to the explanations. However, to be certain what actually causes the inconsistencies in Zorro, other values of the hyper parameter $\tau$ would need to be investigated, which was not done due to time limitations. 

%The higher importance scores of Baseline can thus be interpreted as the \acrshort{gcn} model being more robust in the sense that it is not as susceptible to noise. This indicates that there might be some differences between the two models, even if their performance is roughly the same. The differences between the models in the analysis may indicate that although they utilise the same information and have the same performance, the way they make predictions is slightly different. GCN perhaps obtains a more robust prediction that is not as susceptible to noise because it can formulate the same information Baseline can use in the context of graph structure. 

Having discussed the inconsistencies regarding the Zorro results, the small but nevertheless present differences between the naive node removal and Zorro methods needs to be addressed. These differences may perhaps be reduced by another choice of $\tau$, since another $\tau$ affects the result of Zorro. Even so it is still important to remember that the two methods answers slightly different questions. Naive node removal answers the question which removed node affect the performance of the models the most, whilst Zorro answers the question which node is most important to unmask for a prediction to be as close to the original as possible. Even if the two questions seem to be very similar the result should not be expected to be identical. However, both questions still indicates how important nodes are for the predictions, and using two different methods can thus strengthen the credibility of the results. Extending the analysis to include more methods could thus be beneficial, but this is left as future work.


With the two analysis methods discussed, we can now re-examine the analysis results in this new light. Comparing the methods further strengthens the interpretation that \acrshort{vv1m} for sex and \acrshort{cb1}, \acrshort{dmn} and \acrshort{sn} for age are not among the most important nodes. Furthermore, comparison yields that \acrshort{smm} is most important for sex classification followed by \acrshort{cb2} and \acrshort{pl}. For age prediction, comparing the two methods for both models clearly indicates that \acrshort{smm} and \acrshort{cb2} are the most important nodes and that \acrshort{pmc} also may be somewhat important. Interestingly, \acrshort{smm} and \acrshort{cb2} are thus deemed important for both sex and age prediction. This overlap may be expected, since sex and age are covariates \cite{zhang_covariates}, in the sense that changes in functional connectivity of the brain that are related to age may also be related to sex. In this work we have focused on the effects of sex and ageing in general, without controlling for the other. This may thus be a limitation in the sense that the analysis e.g. compounds the effects of sex-related differences in functional connectivity when studying ageing and vice versa. More in-depth analysis in which sex and age are controlled for would thus be a suitable direction for future work.

Comparing our results with the literature, we find both similarities and inconsistencies. For sex prediction, \cite{understanding_gnn} performed a saliency mapping using similar machine learning models as in this thesis, and arrived at the \acrshort{dmn} and \acrlong{smn} (\acrshort{smn}) being the most important for sex prediction. In \cite{arslan}, the \acrshort{dmn} was found to be important. For age prediction, Song et al. \cite{song_reorganizational} found the \acrshort{dmn} and \acrshort{smn} to undergo relatively large changes in connectivity with age. Tomasi et al. \cite{tomasi_aging} studied the connections between networks and found, among other nodes, that the \acrshort{dmn}, the Dorsal Attention Network (DAN), the \acrlong{ssn} (\acrshort{ssn}) and Cerebellum exhibited pronounced changes in functional connectivity with age. A machine learning approach using \acrshort{svm}s was used in Meier et al. \cite{meier_svm}, where the \acrshort{dmn}, \acrshort{smn} and cingulo-opercular networks were found to be related to age. Most of these works are to some extent in agreement with our results of what networks are important for predicting sex and brain age. The \acrshort{smn}, which includes the \acrshort{smm} that we find important, is often found to be among the most important networks for both sex and age, and the cerebellum is obtained in \cite{tomasi_aging}. However, one major inconsistency is that all of the studied articles find the \acrshort{dmn} to be among the most important networks. 

One possible explanation for the inconsistencies regarding the \acrshort{dmn} may be the dimensionality of the data. In several of the studies mentioned higher dimensional data have been used in which the \acrshort{dmn} and other networks consist of several nodes. It is possible that the useful information for sex or age prediction resides within the \acrshort{dmn} or in how individual parts of the \acrshort{dmn} are connected to other networks. This could explain both the fact that we obtain slightly less performance than for example \cite{arslan, understanding_gnn} and why \acrshort{dmn} is not considered as important as other networks in our thesis. It is still important, however, to remember that we find all networks to be important including \acrshort{dmn}. Several of the studies which have observed age related changes in \acrshort{dmn} \cite{meier_svm, song_reorganizational, tomasi_aging} have no way of ranking the networks, so our observation of \acrshort{smm} and \acrshort{cb2} rather than \acrshort{dmn} being the most important does not need to be an inconsistency. In \cite{arslan, understanding_gnn} they however have the ability to rank importance of nodes for sex classification and they find \acrshort{dmn} to be one of the most important while not even mentioning the cerebellum. An explanation to the disagreement can be because of the different methods used to perform the salience mapping. Just as for the naive node masking method and Zorro, the methods in \cite{understanding_gnn, arslan} might answer different question than our methods and thus gives different results. This highlight the complexity of understanding machine learning models, and simply stating that a node is important might be insufficient. To be sure a specific network is related to sex or age the important question to answer is why and how this network is important for the prediction. 

Since the saliency mapping methods used in this thesis are based on investigating the machine learning models, the significance of the analysis results is limited by the performance of the models. The underlying assumption for the analysis to be meaningful is that the machine learning models can extract relevant information for predicting sex and brain age from the data. Thus, any limitations on the model performance is by extension a limitation on the analysis. In this sense, the low resolution of the data used to train the machine learning models hinders the analysis. Because of this, using more highly resolved data is crucial for improving the significance and making the analysis more detailed. However, using higher dimensionality data can make the analysis technically more challenging and interpretation of the results more difficult. From this perspective, one could argue that a decrease in model performance is worth it for the sake of interpretability of the results. Interpretability is important both for research purposes, but also for clinical applications. Specifically, the 21 node data used in this thesis has the benefit of each node roughly corresponding to a functional brain network, which greatly simplifies the interpretation of the analysis. Another benefit of using the 21 node data is that, as our results indicate, the use of more complex models is not necessary. Using simpler models, such as the Baseline regression model, makes the models themselves more explainable. Thus, there exists a clear trade-off between model and analysis performance and interpretability. Finding the sweet spot between these two is key to further gain insight into how functional brain networks are related to sex and ageing using machine learning. 


